
\documentclass[norsk, a4paper]{report}

\usepackage[utf8x]{inputenc}
\usepackage{babel}
\usepackage{authblk}
\usepackage{longtable}
\usepackage{multicol}
\usepackage{color}
\usepackage{wasysym}
\usepackage[raggedright]{titlesec}
\usepackage{color,hyperref}

% added by reinhard.seifert@helse-bergen.no for NORIC:
\usepackage{booktabs}
\usepackage{rotating}
\usepackage{Sweave}







% Gjør alle lenker mørkeblå
\definecolor{darkblue}{rgb}{0.0,0.0,0.3}
\hypersetup{colorlinks,breaklinks,linkcolor=darkblue,urlcolor=darkblue,anchorcolor=darkblue,citecolor=darkblue}

% Ikke sett inn sidenummer for ``Parts''
\makeatletter
\let\sv@endpart\@endpart
\def\@endpart{\thispagestyle{empty}\sv@endpart}
\makeatother

% Ikke bruk innrykk for nye avsnitt, men heller litt større vertikal avatand
\setlength{\parindent}{0pt}
\setlength{\parskip}{1ex plus 0.5ex minus 0.2ex}


% Sett registerets navn
\def \registernavn {\textit{Norsk register for invasiv kardiologi (NORIC)}}


% tittelting
\title{\registernavn \\ \textbf{Årsrapport for \textit{2013} til offentliggjøring}}

\author[$\dag$]{Siren Hovland\thanks{Daglig leder: \texttt{siren.hovland@helse-bergen.no}}}
\author[$\dag$]{Reinhard Seifert\thanks{Statistiker: \texttt{reinhard.seifert@helse-bergen.no}}}
\author[$\dag$]{Janne Dyngeland\thanks{Registerkoordinator: \texttt{janne.dyngeland@helse-bergen.no}}}
\author[$\dag$]{Svein Rotevatn\thanks{Faglig leder: \texttt{svein.rotevatn@helse-bergen.no}}}
\affil[$\dag$]{NORIC\\Norsk register for invasiv kardiologi \\
Haukeland universitetssjukehus \\
Hjerteavdelingen \\
Postboks 1400 \\
Jonas Lies vei 65 \\
5021 Bergen \\
epost \texttt{noric@helse-bergen.no} \\
Telefon 55 97 22 20}

\renewcommand\Authands{ og }
\renewcommand\Authfont{\scshape}
\renewcommand\Affilfont{\itshape\small}





% overstyre mulige ord-deling-er
\hyphenation{stadium-vurder-ing system-atiske pasi-ent-rap-porterte
retnings-linjer hjemmelsgrunnlag data-behandlings-ansvar nasjon-ale
indi-kator-er styrings-gruppe}



\begin{document}

\maketitle


\tableofcontents


\part{Årsrapport}\label{par:rap}
\thispagestyle{empty}



\chapter{Formålet til registeret}
Norsk register for invasiv kardiologi (NORIC) inngår som et av flere kvalitetsregistre knyttet til Hjerte- og karregisteret ved Folkehelseinstituttet og omfatter invasive kardiologiske prosedyrer. Formålet for registeret er derfor regulert av forskriften til Hjerte- og karregisteret; ''å bidra til bedre kvalitet på helsehjelpen til personer med hjerte- og karsykdommer. Opplysninger i registeret skal benyttes til forebyggende arbeid, kvalitetsforbedring og helseforskning. Registeret skal også utgjøre et grunnlag for styring og planlegging av helsetjenester rettet mot personer med hjerte- og karsykdommer, overvåkning av nye tilfeller og forekomst av slike sykdommer i befolkningen''.

\chapter{Pasientgruppen som omfattes av registeret}
Pasienter som får utført invasive kardiologiske prosedyrer ved sykehus i Norge er omfattet av registeret. Det innbefatter i første rekke undersøkelser og behandlingsformer som koronar angiografi og PCI der tilgangen til hjertet er perkutan, til forskjell fra åpen kirurgisk tilgang. Registeret har også moduler for perkutan hjerteklaff-behandling og for koronar CT-angiografi. Registeret omfatter ikke elektrofysiologiske undersøkelser og behandlinger, og ikke pacemakerbehandling.


\chapter{Oversikt over sykehusene som behandler pasientgruppen}
Tabellen \ref{tab:oppstart} viser hvilke sykehus som behandler denne pasientgruppen. De to første startet registreringer i NORIC i 2013 og ytterligere 4 andre fra januar 2014.  Rikshopsitalet og Feiringklinikken har ennå ikke begynt med registreringer.

\begin{table}[ht]
\centering
\begin{tabular}{lr}
    \toprule
    Sykehus & Oppstart dato\\
    \midrule
    Haukeland universitetssjukehus & 01.01.2013\\
    Universitetssykehuset Nord-Norge & 01.05.2013\\
    St. Olavs hospital & 01.01.2014\\
    Stavanger universitetssjukehus & 01.01.2014\\
    Sørlandet sykehus, Arendal & 01.01.2014\\
    Oslo universitetssykehus, Ullevål & 01.01.2014\\
    Oslo universitetssykehus, Rikshospitalet & Ikke startet ennå\\
    Feiringklinikken & Ikke startet ennå\\
    \bottomrule
\end{tabular}
\label{tab:oppstart}
\caption{Oversikt over deltakende sykehus/senter og oppstarts dato}
\end{table}

\clearpage



\chapter{De viktigste resultatene}

\section{Innledning}

Resultatrapporteringen omfatter bare data fra Universitetssykehuset Nord-Norge (UNN) og Haukeland universitetssjukehus (HUS) for 2013. Analysene er utarbeidet i Rapporteket og er så langt nokså grunnleggende. 
Enkle tabeller og figurer omtaler totalt antall og prosentandel av gjennomførte angiografier og PCIer etter det enkelte sykehus (PCI senter) med følgende utvalgte registreringskategorier:

\begin{itemize}
  \item{Prosedyretype}
    \begin{itemize}
      \item{Angio}
      \item{PCI}
      \item{Angio+PCI}
    \end{itemize}

Angio innebærer diagnostisk hjertekaterisering med røntgenkontrast. PCI omfatter ulike perkutane intervensjonsmetoder for behandling av stenose/okklusjon i ett eller flere koronarkar. Ved Angio+PCI gjøres intervensjonen i direkte tilknytning til angiografiundersøkelsen.

  \item{Hastegrad}
    \begin{itemize}
      \item{Planlagt}
      \item{Akutt}
      \item{Subakutt}
    \end{itemize}

Planlagte prosedyrer gjelder all elektiv utredning og behandling. Subakutt omfatter de prosedyrer som må utføres før pasienten kan skrives ut fra sykehuset. Med akutt menes prosedyrer som ikke kan utsettes, men må gjennomføres umiddelbart på grunn av pasientens tilstand.

  \item{Vakttid}
    \begin{itemize}
      \item{Subakutt tilfelle på dagtid}
      \item{Subakutt tilfelle på vakttid}
      \item{Akutt tilfelle på dagtid}
      \item{Akutt tilfelle på vakttid}
    \end{itemize}


  \item{Indikasjon}
    \begin{itemize}
      \item{SCAD = Stabil koronarsykdom}
      \item{UAP = Ustabil angina pectoris}
      \item{NSTEMI = non-ST segment elevation myocardial infarction}
      \item{STEMI = ST segment elevation myocardial infarction}
      \item{Annet}
    \end{itemize}


  \item{Primær beslutning:}
    \begin{itemize}
      \item{Ingen behandling}
      \item{Fortsatt medisinsk behandling}
      \item{Diskusjon/uavklart}
      \item{PCI ad hoc}
      \item{Annet}
    \end{itemize}

Den primære beslutningen refererer til angiograførens preliminære og mest sannsynlige beslutning etter avsluttet angiografi. 

Ingen behandling betyr at det ikke er behov for PCI, hjertekirurgi eller medisinsk behandling. PCI ad hoc betyr at det blir utført PCI direkte i tilslutning til angiografi. 

  \item{Funn:}
    \begin{itemize}
      \item{1-kar ikke HS}
      \item{2-kar ikke HS}
      \item{3-kar ikke HS}
      \item{HS}
      \item{HS + 1-kar}
      \item{HS + 2-kar}
      \item{HS + 3-kar}
      \item{Ikke konklusiv}
      \item{Normalt/Ateromatose}
    \end{itemize}
       
1-3-kar referer til hvor mange av koronarkarene som har stenoser (innsnevringer).
HS er stenose i venstre hovedstamme. 


  \item{Annen diagnostikk ved angiografi:}
    \begin{itemize}
      \item{FFR = Fractional flow reserve}
      \item{IVUS = intravaskulær ultralyd}
    \end{itemize}

FFR refererer til måling av intrakoronart trykk under infusjon med Adenosin og utføres for å vurdere hvorvidt en stenose er signifikant eller ikke. IVUS nyttes ved behov for ytterligere billeddiagnostikk i forbindelse med angiografi og PCI.

\end{itemize}

Manglende registreringer er i alle tabeller oppført som et enkelt mål for registreringskvaliteten. Haukeland universitetssjukehus (HUS) og Universitetssykehuset Nord-Norge (UNN) var de eneste sykehusene som tok i bruk NORIC i 2013, med oppstart fra henholdsvis 01.januar og 1. mai (figur \ref{fig:N:SHUS}). 

Registreringene viser stor grad av samsvar mellom sykehusene når det gjelder andel av pasienter som får utført PCI (41 \%). Andelen av planlagte (elektive) prosedyrer er høyere ved UNN (48,7 \%) enn ved HUS (40,2 \%), mens andelen subakutte prosedyrer er tilsvarende lavere ved UNN (39 \%) sammenlignet med HUS (46 \%). Andelen av pasienter med indikasjon stabil koronarsykdom er høyere ved UNN (38,6 \%) enn ved HUS (29,2 \%), mens andelen ustabil angina er lavere ved UNN (14.4 \%) enn ved HUS (21 \%). Prosentandel av prosedyrer i registrerings-kategoriene var nokså sammenlignbart, med unntak av 9,4 \% høyere andel stabile pasienter ved HUS (tabell \ref{tab:pros:ind}) som samsvarer med en 7,5 \% lavere andel normale funn (tabell \ref{tab:pros:funn}).

Registreringene viser at det er forskjeller i bruk av supplerende intrakoronar diagnostikk ved de to sykehusene, med større bruk av intravaskulær ultralyd (IVUS) og måling av fractional flow reserve (FFR) ved HUS. 

Analyser av ventetid for pasienten med NSTEMI (ikke-ST-elevasjonsinfarkt) viser betydelige forskjeller mellom de to sykehusene med vesentlig lengre ventetid ved HUS enn ved UNN. For pasienter som blir innlagt direkte til disse sykehusene er median ventetid 2 døgn ved HUS og 1 døgn ved UNN.  For pasienter som blir først innlagt ved annet sykehus er median ventetid ved HUS 5 døgn og 3 døgn ved UNN. Spesielt ved HUS er ventetiden for mange pasienter vesentlig lengre enn det som er anbefalt i internasjonale retningslinjer.  



\section{Antall registrerte prosedyrer}

\begin{figure}[ht]
  \centering
  \label{fig:N:SHUS}

\includegraphics[width=0.95\textwidth]{N_procedures_month_PCIcenter.pdf}\caption{Antall prosedyrer etter måned og PCI senter}
\end{figure}



 %%% mcetype

\clearpage
\subsection{\ldots etter hastegrad}
  
\begin{figure}[ht]
  \centering
\includegraphics[width=0.95\textwidth]{N_procedures_mcetype_PCIcenter.pdf}\caption{Antall prosedyrer etter hastegrad og PCI senter}
\end{figure}


% latex table generated in R 3.1.1 by xtable 1.7-4 package
% Mon Oct  6 11:51:58 2014
\begin{table}[ht]
\centering
\begin{tabular}{rrrrrr}
  \toprule
 & \begin{sideways} Planlagt \end{sideways} & \begin{sideways} Akutt \end{sideways} & \begin{sideways} Subakutt \end{sideways} & \begin{sideways} NA \end{sideways} & \begin{sideways} Sum \end{sideways} \\ 
  \midrule
Haukeland universitetssjukehus & 1287 & 444 & 1473 & 0 & 3204 \\ 
  Universitetssykehuset Nord-Norge & 889 & 225 & 713 & 0 & 1827 \\ 
  Sum & 2176 & 669 & 2186 & 0 & 5031 \\ 
   \bottomrule
\end{tabular}
\caption{Antall prosdyrer etter hastegrad og PCI senter} 
\end{table}

\clearpage
\begin{figure}[ht]
  \centering
\includegraphics[width=0.95\textwidth]{r_procedures_mcetype_PCIcenter.pdf}\caption{Prosentandel prosedyrer etter hastegrad og PCI senter}
\end{figure}


\begin{tiny}
% latex table generated in R 3.1.1 by xtable 1.7-4 package
% Mon Oct  6 11:51:58 2014
\begin{table}[ht]
\centering
\begin{tabular}{rrrrr}
  \toprule
 & \begin{sideways} Planlagt \end{sideways} & \begin{sideways} Akutt \end{sideways} & \begin{sideways} Subakutt \end{sideways} & \begin{sideways} NA \end{sideways} \\ 
  \midrule
Haukeland universitetssjukehus & 40.2 & 13.9 & 46.0 & 0.0 \\ 
  Universitetssykehuset Nord-Norge & 48.7 & 12.3 & 39.0 & 0.0 \\ 
   \bottomrule
\end{tabular}
\caption{Prosentandel prosdyrer etter hastegrad og PCI senter} 
\end{table}\end{tiny}



\clearpage
\subsection{\ldots etter type (Angio/PCI)}

   %%% REGTYP

\begin{figure}[ht]
  \centering
\includegraphics[width=0.95\textwidth]{N_procedures_regtyp_PCIcenter.pdf}  \caption{Antall prosedyrer etter type og PCI senter}
\end{figure}


\begin{tiny}
% latex table generated in R 3.1.1 by xtable 1.7-4 package
% Mon Oct  6 11:51:58 2014
\begin{table}[ht]
\centering
\begin{tabular}{rrrrrr}
  \toprule
 & \begin{sideways} Angio \end{sideways} & \begin{sideways} PCI \end{sideways} & \begin{sideways} Angio+PCI \end{sideways} & \begin{sideways} NA \end{sideways} & \begin{sideways} Sum \end{sideways} \\ 
  \midrule
Haukeland universitetssjukehus & 1914 & 169 & 1121 & 0 & 3204 \\ 
  Universitetssykehuset Nord-Norge & 1068 & 69 & 690 & 0 & 1827 \\ 
  Sum & 2982 & 238 & 1811 & 0 & 5031 \\ 
   \bottomrule
\end{tabular}
\caption{Antall prosdyrer etter type og PCI senter} 
\end{table}\end{tiny}

\clearpage



\begin{figure}[ht]
  \centering
\includegraphics[width=0.95\textwidth]{r_procedures_regtyp_PCIcenter.pdf}  \caption{Prosentandel prosedyrer etter type og PCI senter}
\end{figure}

\begin{tiny}
% latex table generated in R 3.1.1 by xtable 1.7-4 package
% Mon Oct  6 11:51:58 2014
\begin{table}[ht]
\centering
\begin{tabular}{rrrrr}
  \toprule
 & \begin{sideways} Angio \end{sideways} & \begin{sideways} PCI \end{sideways} & \begin{sideways} Angio+PCI \end{sideways} & \begin{sideways} NA \end{sideways} \\ 
  \midrule
Haukeland universitetssjukehus & 59.7 & 5.3 & 35.0 & 0.0 \\ 
  Universitetssykehuset Nord-Norge & 58.5 & 3.8 & 37.8 & 0.0 \\ 
   \bottomrule
\end{tabular}
\caption{Prosentandel prosdyrer etter type og PCI senter} 
\end{table}\end{tiny}



\clearpage
\subsection{\ldots etter vakttid}


   %%% JOURTID

\begin{figure}[ht]
  \centering
\includegraphics[width=0.95\textwidth]{N_procedures_jourtid_PCIcenter.pdf}  \caption{Antall prosdyrer etter vakttid og PCI senter}
\end{figure}

\begin{tiny}
% latex table generated in R 3.1.1 by xtable 1.7-4 package
% Mon Oct  6 11:51:58 2014
\begin{table}[ht]
\centering
\begin{tabular}{rrrrrrrr}
  \toprule
 & \begin{sideways} Planlagt på dagtid \end{sideways} & \begin{sideways} Akuttilfelle på dagtid \end{sideways} & \begin{sideways} Akuttilfelle på vakttid \end{sideways} & \begin{sideways} Subakuttilfelle på dagtid \end{sideways} & \begin{sideways} Subakuttilfelle på vakttid \end{sideways} & \begin{sideways} NA \end{sideways} & \begin{sideways} Sum \end{sideways} \\ 
  \midrule
Haukeland universitetssjukehus & 1280 & 190 & 219 & 1272 & 98 & 145 & 3204 \\ 
  Universitetssykehuset Nord-Norge & 886 & 64 & 122 & 412 & 116 & 227 & 1827 \\ 
  Sum & 2166 & 254 & 341 & 1684 & 214 & 372 & 5031 \\ 
   \bottomrule
\end{tabular}
\caption{Antall prosedyrer etter vakttid og PCI senter} 
\end{table}\end{tiny}

\clearpage


\begin{figure}[ht]
  \centering
\includegraphics[width=0.95\textwidth]{r_procedures_jourtid_PCIcenter.pdf}  \caption{Prosentandel prosdyrer etter vakttid og PCI senter}
\end{figure}

\begin{tiny}
% latex table generated in R 3.1.1 by xtable 1.7-4 package
% Mon Oct  6 11:51:58 2014
\begin{table}[ht]
\centering
\begin{tabular}{rrrrrrr}
  \toprule
 & \begin{sideways} Planlagt på dagtid \end{sideways} & \begin{sideways} Akuttilfelle på dagtid \end{sideways} & \begin{sideways} Akuttilfelle på vakttid \end{sideways} & \begin{sideways} Subakuttilfelle på dagtid \end{sideways} & \begin{sideways} Subakuttilfelle på vakttid \end{sideways} & \begin{sideways} NA \end{sideways} \\ 
  \midrule
Haukeland universitetssjukehus & 40.0 & 5.9 & 6.8 & 39.7 & 3.1 & 4.5 \\ 
  Universitetssykehuset Nord-Norge & 48.5 & 3.5 & 6.7 & 22.6 & 6.3 & 12.4 \\ 
  Sum & 43.1 & 5.0 & 6.8 & 33.5 & 4.3 & 7.4 \\ 
   \bottomrule
\end{tabular}
\caption{Prosentandel prosedyrer etter vakttid og PCI senter} 
\end{table}\end{tiny}



\clearpage
\subsection{\ldots etter indikasjon}

   %%% INDIKATION

\begin{figure}[ht]
  \centering
\includegraphics[width=0.95\textwidth]{N_procedures_indikation_PCIcenter.pdf}  \caption{Antall prosedyrer etter indikasjon og PCI senter (SCAD = stabil koronarsykdom, UAP = ustabil angina pectoris, NSTEMI = non-ST-elevasjon myokardinfarkt, STEMI = ST-elevasjon myokardinfarkt)}
\end{figure}

\begin{tiny}
% latex table generated in R 3.1.1 by xtable 1.7-4 package
% Mon Oct  6 11:51:58 2014
\begin{table}[ht]
\centering
\begin{tabular}{rrrrrrrr}
  \toprule
 & \begin{sideways} SCAD \end{sideways} & \begin{sideways} UAP \end{sideways} & \begin{sideways} NSTEMI \end{sideways} & \begin{sideways} STEMI \end{sideways} & \begin{sideways} Annet \end{sideways} & \begin{sideways} NA \end{sideways} & \begin{sideways} Sum \end{sideways} \\ 
  \midrule
Haukeland universitetssjukehus & 936 & 674 & 757 & 335 & 471 & 31 & 3204 \\ 
  Universitetssykehuset Nord-Norge & 705 & 263 & 399 & 127 & 324 & 9 & 1827 \\ 
  Sum & 1641 & 937 & 1156 & 462 & 795 & 40 & 5031 \\ 
   \bottomrule
\end{tabular}
\caption{Antall prosedyrer etter indikasjon og PCI senter} 
\end{table}\end{tiny}

\clearpage



\begin{figure}[ht]
  \centering
\includegraphics[width=0.95\textwidth]{r_procedures_indikation_PCIcenter.pdf}  \caption{Prosentandel prosedyrer etter indikasjon og PCI senter (SCAD = stabil koronarsykdom, UAP = ustabil angina pectoris, NSTEMI = non-ST-elevasjon myokardinfarkt, STEMI = ST-elevasjon myokardinfarkt)}
\end{figure}

\begin{tiny}
% latex table generated in R 3.1.1 by xtable 1.7-4 package
% Mon Oct  6 11:51:58 2014
\begin{table}[ht]
\centering
\begin{tabular}{rrrrrrr}
  \toprule
 & \begin{sideways} SCAD \end{sideways} & \begin{sideways} UAP \end{sideways} & \begin{sideways} NSTEMI \end{sideways} & \begin{sideways} STEMI \end{sideways} & \begin{sideways} Annet \end{sideways} & \begin{sideways} NA \end{sideways} \\ 
  \midrule
Haukeland universitetssjukehus & 29.2 & 21.0 & 23.6 & 10.5 & 14.7 & 1.0 \\ 
  Universitetssykehuset Nord-Norge & 38.6 & 14.4 & 21.8 & 7.0 & 17.7 & 0.5 \\ 
  Sum & 32.6 & 18.6 & 23.0 & 9.2 & 15.8 & 0.8 \\ 
   \bottomrule
\end{tabular}
\caption{Prosentandel prosedyrer etter indikasjon og PCI senter} 
\label{tab:pros:ind}
\end{table}\end{tiny}



\clearpage
\subsection{\ldots etter primær beslutning}

   %%% PRIMBES

\begin{figure}[ht]
  \centering
\includegraphics[width=0.95\textwidth]{N_procedures_primbes_PCIcenter.pdf}  \caption{Antall prosedyrer etter primær beslutning og PCI senter}
\end{figure}

\begin{tiny}
% latex table generated in R 3.1.1 by xtable 1.7-4 package
% Mon Oct  6 11:51:58 2014
\begin{table}[ht]
\centering
\begin{tabular}{rrrrrrrrr}
  \toprule
 & \begin{sideways} Ingen behandling \end{sideways} & \begin{sideways} Fortsatt medisinsk behandling \end{sideways} & \begin{sideways} Diskusjon/uavklart \end{sideways} & \begin{sideways} PCI elektiv \end{sideways} & \begin{sideways} PCI ad hoc \end{sideways} & \begin{sideways} Annet \end{sideways} & \begin{sideways} NA \end{sideways} & \begin{sideways} Sum \end{sideways} \\ 
  \midrule
Haukeland universitetssjukehus & 136 & 1012 & 676 & 65 & 1115 & 2 & 198 & 3204 \\ 
  Universitetssykehuset Nord-Norge & 179 & 474 & 363 & 28 & 686 & 19 & 78 & 1827 \\ 
  Sum & 315 & 1486 & 1039 & 93 & 1801 & 21 & 276 & 5031 \\ 
   \bottomrule
\end{tabular}
\caption{Antall prosedyrer etter primær beslutning og PCI senter} 
\end{table}\end{tiny}

\clearpage


\begin{figure}[ht]
  \centering
\includegraphics[width=0.95\textwidth]{r_procedures_primbes_PCIcenter.pdf}  \caption{Prosentandel prosedyrer etter primær beslutning og PCI senter}
\end{figure}

\begin{tiny}
% latex table generated in R 3.1.1 by xtable 1.7-4 package
% Mon Oct  6 11:51:58 2014
\begin{table}[ht]
\centering
\begin{tabular}{rrrrrrrr}
  \toprule
 & \begin{sideways} Ingen behandling \end{sideways} & \begin{sideways} Fortsatt medisinsk behandling \end{sideways} & \begin{sideways} Diskusjon/uavklart \end{sideways} & \begin{sideways} PCI elektiv \end{sideways} & \begin{sideways} PCI ad hoc \end{sideways} & \begin{sideways} Annet \end{sideways} & \begin{sideways} NA \end{sideways} \\ 
  \midrule
Haukeland universitetssjukehus & 4.2 & 31.6 & 21.1 & 2.0 & 34.8 & 0.1 & 6.2 \\ 
  Universitetssykehuset Nord-Norge & 9.8 & 25.9 & 19.9 & 1.5 & 37.5 & 1.0 & 4.3 \\ 
  Sum & 6.3 & 29.5 & 20.7 & 1.8 & 35.8 & 0.4 & 5.5 \\ 
   \bottomrule
\end{tabular}
\caption{Prosentandel prosedyrer etter primær beslutning og PCI senter} 
\end{table}\end{tiny}



\clearpage
\subsection{\ldots etter funn}


   %%% FYND

\begin{figure}[ht]
  \centering
\includegraphics[width=0.95\textwidth]{N_procedures_fynd_PCIcenter.pdf}  \caption{Antall prosedyrer etter funn og PCI senter (HS = hovedstamme stenose)}
\end{figure}

\begin{tiny}
% latex table generated in R 3.1.1 by xtable 1.7-4 package
% Mon Oct  6 11:51:58 2014
\begin{table}[ht]
\centering
\begin{tabular}{rrrrrrrrrrrr}
  \toprule
 & \begin{sideways} 1-kar ikke HS \end{sideways} & \begin{sideways} 2-kar ikke HS \end{sideways} & \begin{sideways} 3-kar ikke HS \end{sideways} & \begin{sideways} HS \end{sideways} & \begin{sideways} HS + 1-kar \end{sideways} & \begin{sideways} HS + 2-kar \end{sideways} & \begin{sideways} HS + 3-kar \end{sideways} & \begin{sideways} Ikke konklusiv \end{sideways} & \begin{sideways} Normalt/Ateromatose \end{sideways} & \begin{sideways} NA \end{sideways} & \begin{sideways} Sum \end{sideways} \\ 
  \midrule
Haukeland universitetssjukehus & 846 & 554 & 614 & 13 & 37 & 77 & 167 & 1 & 859 & 36 & 3204 \\ 
  Universitetssykehuset Nord-Norge & 493 & 289 & 270 & 8 & 30 & 29 & 71 & 1 & 627 & 9 & 1827 \\ 
  Sum & 1339 & 843 & 884 & 21 & 67 & 106 & 238 & 2 & 1486 & 45 & 5031 \\ 
   \bottomrule
\end{tabular}
\caption{Antall prosedyrer etter funn og PCI senter} 
\end{table}\end{tiny}

\clearpage



\begin{figure}[ht]
  \centering
\includegraphics[width=0.95\textwidth]{r_procedures_fynd_PCIcenter.pdf}  \caption{Prosentandel prosedyrer etter funn og PCI senter (HS = hovedstamme stenose)}
\end{figure}

\begin{tiny}
% latex table generated in R 3.1.1 by xtable 1.7-4 package
% Mon Oct  6 11:51:58 2014
\begin{table}[ht]
\centering
\begin{tabular}{rrrrrrrrrrr}
  \toprule
 & \begin{sideways} 1-kar ikke HS \end{sideways} & \begin{sideways} 2-kar ikke HS \end{sideways} & \begin{sideways} 3-kar ikke HS \end{sideways} & \begin{sideways} HS \end{sideways} & \begin{sideways} HS + 1-kar \end{sideways} & \begin{sideways} HS + 2-kar \end{sideways} & \begin{sideways} HS + 3-kar \end{sideways} & \begin{sideways} Ikke konklusiv \end{sideways} & \begin{sideways} Normalt/Ateromatose \end{sideways} & \begin{sideways} NA \end{sideways} \\ 
  \midrule
Haukeland universitetssjukehus & 26.4 & 17.3 & 19.2 & 0.4 & 1.2 & 2.4 & 5.2 & 0.0 & 26.8 & 1.1 \\ 
  Universitetssykehuset Nord-Norge & 27.0 & 15.8 & 14.8 & 0.4 & 1.6 & 1.6 & 3.9 & 0.1 & 34.3 & 0.5 \\ 
  Sum & 26.6 & 16.8 & 17.6 & 0.4 & 1.3 & 2.1 & 4.7 & 0.0 & 29.5 & 0.9 \\ 
   \bottomrule
\end{tabular}
\caption{Prosentandel prosedyrer etter funn og PCI senter} 
\label{tab:pros:funn}
\end{table}\end{tiny}



\clearpage
\section{Antall annen diagnostikk}

% latex table generated in R 3.1.1 by xtable 1.7-4 package
% Mon Oct  6 11:51:58 2014
\begin{table}[ht]
\centering
\begin{tabular}{rrrrr}
  \toprule
 & \begin{sideways} Antall FFR \end{sideways} & \begin{sideways} Prosentandel FFR \end{sideways} & \begin{sideways} Antall IVUS \end{sideways} & \begin{sideways} Prosentandel IVUS \end{sideways} \\ 
  \midrule
Haukeland universitetssjukehus & 220 & 6.9 & 75 & 2.3 \\ 
  Universitetssykehuset Nord-Norge & 81 & 4.4 & 9 & 0.5 \\ 
   \bottomrule
\end{tabular}
\caption{Totalt antall annen diagnostikk etter PCI senter (prosentandel per totalt antall gjennomførte prosedyrer).} 
\end{table}



\section{Tid fra innleggelse i sykehus til prosedyre}

% latex table generated in R 3.1.1 by xtable 1.7-4 package
% Mon Oct  6 11:51:58 2014
\begin{table}[ht]
\centering
\begin{tabular}{rrrr}
  \toprule
 & 25\% & 50\% & 75\% \\ 
  \midrule
Haukeland universitetssjukehus / Henvist & 3.0 & 5.0 & 7.0 \\ 
  Universitetssykehuset Nord-Norge / Henvist & 2.0 & 3.0 & 5.0 \\ 
  Haukeland universitetssjukehus / Direkte & 1.0 & 2.0 & 4.5 \\ 
  Universitetssykehuset Nord-Norge / Direkte & 0.0 & 1.0 & 1.0 \\ 
   \bottomrule
\end{tabular}
\caption{Percentiler (25\%, 50\% og 75\% ) av antall dager fra innleggelse i sykehus til prosedyre etter henvisnings status og PCI senter. NB: I denne tidlige versjonen av rapportsystemet er ventetider >30 dager eksludert (n=11 fra UNN og n=13 fra HUS. Alle n=24 viser inkonsistente dato-registreringer og må korrigeres manuellt). Siden quartiler og median er robust mot uteliggende feilregistreringer. Tallene er rapportert uten en manuell kvalitetskontroll og er sansynligvis for konservative.} 
\end{table}




\chapter{Dekningsgrad på enhet og individnivå}
NORIC er ikke koblet til basisiregisteret ennå og har ikke kunnet gjennomføre dekningsgradsanalyser på indivinivå.  På institusjonsnivå er det 95 – 97 \% dekningsgrad målt mot dataprogram (Orbit) som blir brukt til planlegging av laboratoriedriften.

\chapter{Hvordan resultatene har blitt benyttet i kvalitetsforbedringsarbeid}
Analyser som er utført for Haukeland universitetssjukehus har vist at ventetiden for angiografi/PCI ved NSTEMI (ikke-ST-elevasjonsinfarkt) er vesentlig lengre enn anbefalt i internasjonalt aksepterte retningslinjer. Dette gjelder i særlig grad pasienter som henvises fra andre sykehus. Ledelsen ved aktuell avdeling er gjort oppmerksom på problemet med lang ventetid ved NSTEMI. Det blir forsøkt å prioritere denne type pasienter til slike undersøkelser, og det vil bli satt ned en gruppe for å vurdere flere tiltak.







\chapter{Fagutvikling og kvalitetsforbedring av tjenesten}

I løpet av 2014 vil vi bestemme nye kvalitetsmål for registeret. I den forbindelse vil vi også se på de kvalitetsmål som blir brukt i Sverige på tilsvarende register der.  Disse er delvis basert på etterlevelse av aksepterte guidelines (retningslinjer) for behandling av spesifikke tilstander.   

Vi ønsker å utvikle system for pasientrapporterte resultater, men er avhengig av finansiering for å gjennomføre dette. Vi vil også bidra til utvikling av nasjonale kvalitetsindikatorer innenfor invasiv kardiologi.

Våre muligheter til å registrere demografiske og sosiale variabler er begrenset siden disse ikke er hjemlet i forskrift for Hjerte- og karregisteret.


\chapter{Formidling av resultater}

Vi har til nå laget en månedsrapport for virksomheten og en rapport som viser forbruket av forskjellige stenter. Vi vil videreutvikle rapportene til fagmiljøene med bl.a. ulike prosessmål og en rapport som viser stråledoser. Vi tar også sikte på å utvikle rapporter for den enkelte operatør, slik at vedkommende kan sjekke sin praksis med et gjennomsnitt for landet.  

Noen av disse rapportene vil kunne finnes på web-siden til registeret og dermed tilgjengelig for pasienter og andre interesserte.



\end{document}
